\section{Conclusioni}
La nostra implementazione ha prestazioni che si avvicinano molto a quelle dell'implementazione built-in di \texttt{MATLAB}, e teoriche.

Tempi di convergenza e numero di iterazioni sono dello stesso ordine di grandezza e anche l'accuratezza della soluzione è la medesima.

Più in particolare PDIP-GMRES risulta l'implementazione più lenta, principalmente perchè contiene la risoluzione iterativa dell'\textit{augmented KKT}; tuttavia le prestazioni di questo metodo migliorano significamene all'aumentare del numero di vincoli, e quindi di righe, in $A$; questo accade perché la struttura della matrice nel nostro problema induce una diminuzione della densità della matrice stessa, all'aumentare di $m$. E GMRES funziona meglio su matrici sparse.
    
Invece PDIP-LDL ha prestazioni sempre paragonabili a QUADPROG: è più lento in media di un fattore $\approx\times2$, soffre più di QUADPROG l'aumento del numero di vincoli, ma è molto più stabile all'aumentare della densità della matrice $Q$.
    
Infine, QUADPROG necessita di circa la metà delle iterazioni per convergere rispetto alle nostre implementazioni, a prescindere dai parametri.
