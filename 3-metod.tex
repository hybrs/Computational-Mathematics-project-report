\section{Metodo Risolutivo}
\subsection{Primal-Dual Interior Point method}
L'intuizione principale dei metodi Primal-Dual è 
quella di considerare sia il problema di minimizzazione
 $P$ che il suo duale $D$ per ottenere un limite 
superiore $v(P)$ ed inferiore $v(D)$ della soluzione.
Da $P$ e $D$ si ricava quindi il sistema KKT \ref{eq:pkkt2} associato e una misura
 della distanza della soluzione attuale dall'ottimo, detta \textit{complementary gap};
  definita come la differenza fra il valore della funzione obiettivo in $P$ e in $D$, eventualmente normalizzata.
  Ad ogni iterazione, una volta trovata una soluzione per \ref{eq:pkkt2}, calcoliamo una nuova coppia di soluzioni primali/duali 
  e riduciamo il complementary gap.

  Con gli elementi presentati finora possiamo delineare la struttura dello pseudocodice del nostro metodo come segue:

\begin{algorithm}
\caption{pseudocodice Interior-Point Primal-Dual method}\label{alg:pseudo}
\begin{algorithmic}[1]
\Function{PDIP}{Q, q, A, b, eps, maxit}
\State inizializza $(\textit{x}_0, \textit{\lambda}_{eq}, \textit{\lambda}_{s}) \geq 0 random
\BState \textbf{while} \emph{true}:
\If {$\textit{string}(i) = \textit{path}(j)$}
\State $j \gets j-1$.
\State $i \gets i-1$.
\State \textbf{goto} \emph{loop}.
\State \textbf{close};
\EndIf
\State $i \gets i+\max(\textit{delta}_1(\textit{string}(i)),\textit{delta}_2(j))$.
\State \textbf{goto} \emph{top}.
\EndFunction
\end{algorithmic}
\end{algorithm}